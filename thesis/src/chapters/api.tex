\definecolor{dkgreen}{rgb}{0,0.6,0}
\definecolor{gray}{rgb}{0.5,0.5,0.5}
\definecolor{mauve}{rgb}{0.58,0,0.82}
\definecolor{backcolour}{rgb}{1,1,1}

\lstset{frame=none,
    backgroundcolor=\color{backcolour}, 
  language=Python,
  aboveskip=5mm,
  belowskip=5mm,
  showstringspaces=false,
  columns=flexible,
  basicstyle={\linespread{0.8}\small\ttfamily},
  numbers=none,
  numberstyle=\tiny\color{gray},
  keywordstyle=\color{dkgreen},
  commentstyle=\color{blue},
  stringstyle=\color{mauve},
  breaklines=false,
  breakatwhitespace=true,
  tabsize=3
}

\chapter{Millipyde API}

This section will explain the Millipyde API that is exposed to Python developers who use the framework. The following are the imports used in our API example, including the `millipyde' package itself that has been imported as `mp.'

\begin{lstlisting}
import numpy as np
from skimage.io import imsave, imread
import millipyde as mp
\end{lstlisting}

%========================================================================================================================================
\section{gpuarray}

\subsection{Construction}

\begin{description}
   \item[Description] Creates a gpu-compatible array that is inter-operable with NumPy \verb|ndarray|s other \verb|ndarray| compatible types and libraries. 
   \item[Parameters] An array-compatible type. Including but not limited to Numpy \verb|ndarrays|, SciPy \verb|ndimages|, and Python's built-in List Type
   \item[Returns] A \verb|gpuarray| instance
   \item[Raises] \phantom{}
   \begin{itemize}
   \item ValueError if the argument is not array-compatible
   \item ValueError if the array's contents are not a numeric type
   \end{itemize}
   \item[Example] \phantom{}
   \begin{lstlisting}
numpy_array = np.array([1, 2, 3, 4])
gpu_array = mp.gpuarray(numpy_array)

numpy_array2 = np.array([[[1, 2, 3], [4, 5, 6]], 
                         [[1, 2, 3], [4, 5, 6]]])
gpu_array2 = mp.gpuarray(numpy_array2)
                                
gpu_array3 = mp.gpuarray([1, 2, 3, 4])
\end{lstlisting}
\end{description}

\subsection{Methods}

\subsubsection{clone}

\begin{description}
   \item[Description] Clone the \verb|gpuarray|. Creates a deep copy so that all memory is unique. The clone's memory may not be on the same GPU device depending on what device is targeted when the copy is created
   \item[Parameters] None
   \item[Returns] A new \verb|gpuarray|
   \item[Raises] None
   \item[Example] \phantom{}
   \begin{lstlisting}
numpy_array = np.array([1, 2, 3, 4])
gpu_array = mp.gpuarray(numpy_array)
gpu_array2 = gpu_array.clone()
\end{lstlisting}
\end{description}

%========================================================================================================================================

\section{gpuimage}

The \verb|gpuimage| type is a subtype of \verb|gpuarray|. All \verb|gpuarray| functions can accept a \verb|gpuimage| as an argument. The \verb|gpuimage| imposes additional dimensional and type constraints, however, so not all \verb|gpuimage| functions can accept all \verb|gpuarray|s.

\subsection{Construction}

\begin{description}
   \item[Description] Creates a gpu-compatible array representing an image. Construction requires either a 2-dimensional array of float values for greyscale images, or a 3-dimensional array of integer values for colored images. The inner most array dimension in colored images can either be 3 elements wide for RGB images, or 4 elements wide for RGBA images such as PNG files.
   \item[Parameters] An array-compatible type. Including but not limited to NumPy \verb|ndarrays|, SciPy \verb|ndimages|, and Python's built-in List Type.
   \item[Returns] A \verb|gpuarray| instance
   \item[Raises] \phantom{}
   \begin{itemize}
       \item ValueError if the argument is not array-compatible
       \item ValueError if the array's contents are not a numeric type
       \item ValueError if the dimensions don't match greyscale or RGB/RGBA images
   \end{itemize}
   \item[Example] Creating two images in two ways\phantom{}
   \begin{lstlisting}
img = mp.gpuimage(io.imread("images/charlie.png"))

array = np.array([[[64, 76, 32], [22, 65, 64]], 
                  [[12, 53, 43], [33, 56, 42]]])
img2 = mp.gpuimage(array)
\end{lstlisting}
\end{description}

\subsection{Methods}

\subsubsection{rgb2grey}

\begin{description}
   \item[Description] Turns the given rgb or rgba image into a greyscale image represented by floating point values between 0 and 1 for each pixel.
   \item[Parameters] None
   \item[Returns] None
   \item[Raises] None
   \item[Example] Equivalent ways to greyscale a single image
   \begin{lstlisting}
img = mp.gpuimage(io.imread("images/charlie.png"))
img.rgb2grey()
# or
img.rgb2gray()
# or
img.rgba2grey()
# or
img.rgba2gray()
\end{lstlisting}
\end{description}

\subsubsection{transpose}

\begin{description}
   \item[Description] Rotates a \verb|gpuimage| 90\textdegree \ counterclockwise using a transposition operation on the GPU.
   \item[Parameters] None
   \item[Returns] None
   \item[Raises] None
   \item[Example] \phantom{}
   \begin{lstlisting}
img = mp.gpuimage(io.imread("images/charlie.png"))
charlie2.transpose()
\end{lstlisting}
\end{description}

\subsubsection{gaussian}

\begin{description}
   \item[Description] Blur a \verb|gpuimage| using a Gaussian function. A clamping value of 0 is used for calculation involving pixels past the edge of the image.
   \item[Parameters] \phantom{}
   \begin{itemize}
   \item sigma: An integer or float value to use as standard-deviation in the calculation of the convolution kernel
   \end{itemize}
   \item[Returns] None
   \item[Raises] None
   \item[Example] \phantom{}
   \begin{lstlisting}
img = mp.gpuimage(io.imread("images/charlie.png"))
charlie2.gaussian(2)
\end{lstlisting}
\end{description}

\subsubsection{random\_gaussian}

\begin{description}
   \item[Description] Blur a \verb|gpuimage| using a Gaussian function using a random standard deviation in a given range. A clamping value of 0 is used for calculation involving pixels past the edge of the image.
   \item[Parameters] \phantom{}
   \begin{itemize}
   \item min\_sigma: An integer or float value to use as the minimum standard-deviation in the calculation of the convolution kernel
   \item max\_sigma: An integer or float value to use as the maximum standard-deviation in the calculation of the convolution kernel
   \end{itemize}
   \item[Returns] None
   \item[Raises] None
   \item[Example] \phantom{}
   \begin{lstlisting}
img = mp.gpuimage(io.imread("images/charlie.png"))
charlie2.random_gaussian(0, 0.5)
\end{lstlisting}
\end{description}

\subsubsection{brightness}

\begin{description}
   \item[Description] Adjust the brightness of a \verb|gpuimage| by adding the percent change specified by the delta
   \item[Parameters] \phantom{}
   \begin{itemize}
   \item delta: A float value between -1 and 1 for how much to change the brightness by. 
   \end{itemize}
   \item[Returns] None
   \item[Raises] \phantom{}
   \begin{itemize}
       \item ValueError if the delta value is not between -1 and 1 
   \end{itemize}
   \item[Example] \phantom{}
   \begin{lstlisting}
img = mp.gpuimage(io.imread("images/charlie.png"))
img.brightness(.2)
\end{lstlisting}
\end{description}

\subsubsection{random\_brightness}

\begin{description}
   \item[Description] Adjust the brightness of a \verb|gpuimage| by adding the percent change specified by a random delta in the given range
   \item[Parameters] \phantom{}
   \begin{itemize}
   \item min\_delta: A float value between -1 and 1 for the minimum amount to change the brightness by.
   \item max\_delta: A float value between -1 and 1 for the maximum amount to change the brightness by. 
   \end{itemize}
   \item[Returns] None
   \item[Raises] \phantom{}
   \begin{itemize}
       \item ValueError if either the maximum or minimum delta values are not between -1 and 1 
   \end{itemize}
   \item[Example] \phantom{}
   \begin{lstlisting}
img = mp.gpuimage(io.imread("images/charlie.png"))
img.random_brightness(-.2, .5)
\end{lstlisting}
\end{description}

\subsubsection{colorize}

\begin{description}
   \item[Description] Adjust the color of a non-greyscale \verb|gpuimage| multiplying the RGB components by the given multipliers
   \item[Parameters] \phantom{}
   \begin{itemize}
   \item r\_mult: A float value for how much the red value is multiplied by
   \item g\_mult: A float value for how much the green value is multiplied by
   \item b\_mult: A float value for how much the blue value is multiplied by
   \end{itemize}
   \item[Returns] None
   \item[Raises] None
   \item[Example] \phantom{}
   \begin{lstlisting}
img = mp.gpuimage(io.imread("images/charlie.png"))
# Boost the red value and leave green and blue as-is
img.colorize(1.2, 1, 1)
\end{lstlisting}
\end{description}

\subsubsection{random\_colorize}

\begin{description}
   \item[Description] Adjust the color of a non-greyscale gpuimage multiplying the RGB components by random multipliers in the given ranges
   \item[Parameters] \phantom{}
   \begin{itemize}
   \item r\_range: A list of two float values representing a lower and upper bound for how much the red value is multiplied by
   \item g\_range: A list of two float values representing a lower and upper bound for how much the green value is multiplied by
   \item b\_range: A list of two float values representing a lower and upper bound for how much the blue value is multiplied by
   \end{itemize}
   \item[Returns] None
   \item[Raises] None
   \item[Example] \phantom{}
   \begin{lstlisting}
img = mp.gpuimage(io.imread("images/charlie.png"))
img.colorize([.5, 1.5], [.5, 1.5], [.5, 1.5])
\end{lstlisting}
\end{description}

\subsubsection{fliplr}

\begin{description}
   \item[Description] Flip the \verb|gpuimage| left-to-right
   \item[Parameters] None
   \item[Returns] None
   \item[Raises] None
   \item[Example] \phantom{}
   \begin{lstlisting}
img = mp.gpuimage(io.imread("images/charlie.png"))
img.fliplr()
\end{lstlisting}
\end{description}

\subsubsection{rotate}

\begin{description}
   \item[Description] Rotate the \verb|gpuimage| counter-clockwise by the given angle
   \item[Parameters] \phantom{}
   \begin{itemize}
   \item angle: An integer or float value for the rotation angle in degrees
   \end{itemize}
   \item[Returns] None
   \item[Raises] None
   \item[Example] \phantom{}
   \begin{lstlisting}
img = mp.gpuimage(io.imread("images/charlie.png"))
# Rotate 5 degrees counter-clockwise
img.rotate(5)
\end{lstlisting}
\end{description}

\subsubsection{random\_rotate}

\begin{description}
   \item[Description] Rotate the \verb|gpuimage| counter-clockwise by a random angle within the given range
   \item[Parameters] \phantom{}
   \begin{itemize}
   \item min\_angle: An integer or float value for the minimum rotation angle in degrees
   \item max\_angle: An integer or float value for the maximum rotation angle in degrees
   \end{itemize}
   \item[Returns] None
   \item[Raises] None
   \item[Example] \phantom{}
   \begin{lstlisting}
img = mp.gpuimage(io.imread("images/charlie.png"))
# Rotate randomly between 45 and 180 degrees counter-clockwise
img.random_rotate(45, 180)
\end{lstlisting}
\end{description}

\subsubsection{adjust\_gamma}

\begin{description}
   \item[Description] Perform gamma correction on the \verb|gpuimage|
   \item[Parameters] \phantom{}
   \begin{itemize}
   \item gamma: An float value for gamma
   \item gain: A float value for gain that is used as a constant multiplier
   \end{itemize}
   \item[Returns] None
   \item[Raises] None
   \item[Example] \phantom{}
   \begin{lstlisting}
img = mp.gpuimage(io.imread("images/charlie.png"))
# 2nd Power gamma with standard gain
img.adjust_gamma(2, 1)
\end{lstlisting}
\end{description}

\subsubsection{random\_adjust\_gamma}

\begin{description}
   \item[Description] Perform a random gamma correction on the \verb|gpuimage|
   \item[Parameters] \phantom{}
   \begin{itemize}
   \item gamma\_range: A list of two float values representing a lower and upper bound for gamma
   \item gain\_range: A list of two float values representing a lower and upper bound for gain
   \end{itemize}
   \item[Returns] None
   \item[Raises] None
   \item[Example] \phantom{}
   \begin{lstlisting}
img = mp.gpuimage(io.imread("images/charlie.png"))
img.adjust_gamma([2, 3], [1, 1.2])
\end{lstlisting}
\end{description}

\subsubsection{clone}

\begin{description}
   \item[Description] Clone the \verb|gpuimage|. Creates a deep copy so that all memory is unique. The clone's memory may not be on the same GPU device depending on what device is targeted when the copy is created
   \item[Parameters] None
   \item[Returns] A new \verb|gpuimage|
   \item[Raises] None
   \item[Example] \phantom{}
   \begin{lstlisting}
img = mp.gpuimage(io.imread("images/charlie.png"))
img2 = img.clone()
\end{lstlisting}
\end{description}

\subsection{Related Module Functions}

\subsubsection{image\_from\_path}

\begin{description}
   \item[Description] Create a \verb|gpuimage| using the image file located at the given path
   \item[Parameters] \phantom{}
   \begin{itemize}
       \item path: A string path for a specific image file. The file must be an appropriate image format such as PNG, JPEG, TIFF, BPM, etc.
   \end{itemize}
   \item[Returns] A new \verb|gpuimage|
   \item[Raises] None
   \item[Example] \phantom{}
   \begin{lstlisting}
# Load from the local images directory
img = image_from_path("images/charlie.png")
\end{lstlisting}
\end{description}

\subsubsection{images\_from\_path}

\begin{description}
   \item[Description] Create a list of \verb|gpuimages| using the all image files located at the given path
   \item[Parameters] \phantom{}
   \begin{itemize}
       \item path: A string path containing the images to load. Files that are not an appropriate image format will be ignored.
   \end{itemize}
   \item[Returns] A new list of \verb|gpuimages|
   \item[Raises] None
   \item[Example] \phantom{}
   \begin{lstlisting}
# Load from the local images directory
img = images_from_path("images/")
# Paths can also exclude the final slash
img = images_from_path("images")
\end{lstlisting}
\end{description}

%========================================================================================================================================

\section{Device}

The Device type can be used in coordination with Python \verb|with| statements to set which target device is used within a given scope. At exit, the \verb|with| statement will return the target Device back to the one used before the \verb|with| call. When exiting the scope, Millipyde will synchronize with the target Device to make sure all work has finished. In the case of an error with the Device, the device will be reset so that the exception thrown can be caught if desired and the target Device can be used in future calls. These \verb|with| statements can also be nested to use specific devices in specific situations and to safely return back to the previous target device when back in the outer \verb|with| statement's scope.

\subsection{Construction}

\begin{description}
   \item[Description] Creates a scope that uses a specific device ID as the gpu device to target. All types constructed inside this scope will use this target for gpu-based memory allocation. Any functions inside the scope will also be executed on the gpu specified by the target Device.
   \item[Parameters] \phantom{}
   \begin{itemize}
       \item device\_id: An integer representing the device ID to use
   \end{itemize}
   \item[Returns] A Device instance
   \item[Raises] Any errors that occur inside the \verb|with| statement
   \item[Example] \phantom{}
   \begin{lstlisting}
with mp.Device(0):
    # img is created on device 0
    img = mp.gpuimage(io.imread("images/charlie.png"))
    with mp.Device(1):
        # img is transferred to device 1 and greyscaled there
        img.rgb2grey()
    # img is transferred back to device 0 and transposed there
    img.transpose()
\end{lstlisting}
\end{description}

\subsection{Related Module Functions}

\subsubsection{get\_current\_device}

\begin{description}
   \item[Description] Get the target Device for the given scope 
   \item[Parameters] None
   \item[Returns] An integer device ID
   \item[Raises] None
   \item[Example] \phantom{}
   \begin{lstlisting}
with mp.Device(0):
    # Returns 0
    d = mp.get_current_device()
    with mp.Device(1):
        # Returns 1
        d = mp.get_current_device()
\end{lstlisting}
\end{description}

\subsubsection{get\_device\_count}

\begin{description}
   \item[Description] Get the number of recognized devices on the system 
   \item[Parameters] None
   \item[Returns] An integer device ID
   \item[Raises] None
   \item[Example] \phantom{}
   \begin{lstlisting}
count = mp.get_device_count()
for i in range(count):
    with mp.Device(i):
        pass
\end{lstlisting}
\end{description}

%========================================================================================================================================

\section{Operation}

\subsection{Construction}

\begin{description}
   \item[Description] Creates an operation around the given function or class method that can be run later
   \item[Parameters] \phantom{}
   \begin{itemize}
       \item function: A callable such as a function or lambda, or a string representing a method name for instance methods
       \item params...: Variable length parameters to be used when calling 'function'
       \item probability [optional]: A float probability between 0 and 1 (exclusive) that is re-evaluated each time this Operation is used to determine whether the function is called
   \end{itemize}
   \item[Returns] An Operation instance
   \item[Raises] \phantom{}
   \begin{itemize}
       \item ValueError for invalid probabilities
   \end{itemize}
   \item[Example] \phantom{}
   \begin{lstlisting}
def add_two_nums(x, y):
    return x + y

# Operation constructed from function in current scope
op = mp.Operation(add_two_nums, 4, 6)

# Operation constructed from instance method
grey_op = mp.Operation("rgb2grey")

# Operation with arguments and probability
gauss_op = mp.Operation("gaussian", 2, probability=.5)

\end{lstlisting}
\end{description}

\subsection{Methods}

\subsubsection{run}

\begin{description}
   \item[Description] Run the given Operation that represents a function. This method cannot run an Operation representing an instance method. See run\_on for more.
   \item[Parameters] None
   \item[Returns] Return the result of calling the function represented by the Operation, or None if the function did not run at all due to its probability
   \item[Raises] \phantom{}
   \begin{itemize}
       \item RuntimeError if the probability calculation fails. This may be due to a system call failure in the random number retrieval
   \end{itemize}
   \item[Example] \phantom{}
   \begin{lstlisting}
def print_hello():
    print("Hello world!")

hello_op = mp.Operation(print_hello, probability=.8)
operation.run()

\end{lstlisting}
\end{description}

\subsubsection{run\_on}

\begin{description}
   \item[Description] Run the given Operation that represents an instance method.
   \item[Parameters]
   \begin{itemize}
       \item instance: The instance to run the instance method on
   \end{itemize}
   \item[Returns] Return the result of calling the instance method represented by the Operation, or None if the function did not run at all due to its probability
   \item[Raises] \phantom{}
   \begin{itemize}
       \item RuntimeError if the probability calculation fails. This may be due to a system call failure in the random number retrieval
       \item ValueError if the given instance does not have this method
   \end{itemize}
   \item[Example] \phantom{}
   \begin{lstlisting}
img = mp.image_from_path("/images/charlie.png")
gauss_op = mp.Operation("gaussian", 2, probability=.5)

gauss_op.run_on(img)

\end{lstlisting}
\end{description}

%========================================================================================================================================

\section{Pipeline}

\subsection{Construction}

\begin{description}
   \item[Description] Creates a Pipeline
   \item[Parameters] \phantom{}
   \begin{itemize}
       \item inputs: A list of gpu-compatible types (gpuarray and/or gpuimage)
       \item operations: a list of Operation types. All Operatons representing instance methods must be able to operate on gpu-types
       \item device [optional]:  An integer representing the ID of the device that the pipeline will execute on. Overrides any targeted device. Will attempt to use all available devices if none is specified and if no target device is set.
   \end{itemize}
   \item[Returns] A Pipeline instance
   \item[Raises] \phantom{}
   \begin{itemize}
       \item ValueError for invalid parameters
   \end{itemize}
   \item[Example] \phantom{}
   \begin{lstlisting}
inputs = [mp.image_from_path("images/charlie.png"),
          mp.image_from_path("images/aspen.png")]

operations = [mp.Operation("colorize", 1.5, .8. 1),
               mp.Operation("transpose"),
               mp.Operation("rgb2grey", probability=0.5)]

# The following pipeline will attempt to use all available devices
p1 = mp.Pipeline(inputs, operations)

# The following pipeline will only schedule on device 1
with mp.Device(1):
    p2 = mp.Pipeline(inputs, operations)

# The following will only schedule on device 0
# It will override target device 1
with mp.Device(1):
    p3 = mp.Pipeline(inputs, operations, device=0)

\end{lstlisting}
\end{description}

\subsection{Methods}

\subsubsection{connect\_to}

\begin{description}
   \item[Description] Connect one Pipeline to another so that the outputs from one become the inputs to the other. They will execute concurrently as soon as any of the outputs are ready from the first pipeline. Millipyde will attempt to find an optimal way of scheduling each Pipeline on a separate device if one or both Pipelines are not specified to use a specific device
   \item[Parameters] \phantom{}
   \begin{itemize}
       \item pipeline: The Pipeline that will accept this Pipeline's results as inputs
   \end{itemize}
   \item[Returns] None
   \item[Raises] None
   \item[Example] \phantom{}
   \begin{lstlisting}
# Letting Millipyde find the best scheduling of devices it can
p = mp.Pipeline(inputs, operations)
p2 = mp.Pipeline([], operations2)
p.connect_to(p2)

# Forcing both pipelines to use specific devices
p = mp.Pipeline(inputs, operations, device=0)
p2 = mp.Pipeline([], operations2, device=1)
p.connect_to(p2)

\end{lstlisting}
\end{description}

\subsubsection{run}

\begin{description}
   \item[Description] Run the Pipeline instance. This function will only return once all inputs have been operated on, and once all connected Pipelines have completed as well. Running a pipeline mutates the inputs.
   \item[Parameters] None
   \item[Returns] None
   \item[Raises] None
   \item[Example] \phantom{}
   \begin{lstlisting}
p1 = mp.Pipeline(inputs, operations)
p2 = mp.Pipeline([], operations2)
p3 = mp.Pipeline([], operations3)
p1.connect_to(p2)
p2.connect_to(p3)

# Returns once p1, p2, and p3 have completed
p.run()

\end{lstlisting}
\end{description}

%========================================================================================================================================

\section{Generator}

\subsection{Construction}

\begin{description}
   \item[Description] Creates a Generator with the given inputs and operations
   \item[Parameters] \phantom{}
   \begin{itemize}
       \item inputs: A list of gpu-compatible types (gpuarray and/or gpuimage), or a string path that contains images that can be turned into gpuimages
       \item operations: a list of Operation types. All Operations representing instance methods must be able to operate on gpu-types
       \item device [optional]: An integer representing the ID of device that the Generator will execute on. Overrides any targeted device. Will attempt to use the best available device if none is specified and no target was specified for this scope.
       \item outputs [optional]: An integer representing the amount of outputs to create. The output set size will be infinite if none is specified
       \item return\_to\_host [optional]: A boolean value for whether or not the results should be automatically turned into NumPy ndarrays. 
   \end{itemize}
   \item[Returns] A Generator instance
   \item[Raises] \phantom{}
   \begin{itemize}
       \item ValueError for invalid parameters
       \item StopIteration if we have generated all specified outputs
   \end{itemize}
   \item[Example] \phantom{}
   \begin{lstlisting}
operations = [mp.Operation("colorize", 1.5, .8. 1),
               mp.Operation("transpose"),
               mp.Operation("rgb2grey", probability=0.5)]

# Will produce 3 outputs using device 1
g1 = mp.Generator("examples/images", operations, 
        return_to_host=True, outputs=3, device=1)

i = 1
for result in g1:
    imsave("output" + str(i) + ".png", result)
    i += 1
    
# Will produce N number of outputs on the best available device
g1 = mp.Generator("examples/images", operations, return_to_host=True)

for i in range(n):
    result = next(g)
    imsave("output" + str(i) + ".png", result)

\end{lstlisting}
\end{description}
