The prevalence of general-purpose GPU computing continues to grow and tackle a wider variety of problems that benefit from GPU-acceleration. This acceleration often suffers from a high barrier to entry, however, due to the complexity of software tools that closely map to the underlying GPU hardware, the fast-changing landscape of GPU environments, and the fragmentation of tools and languages that only support specific platforms. Because of this, new solutions will continue to be needed to make GPGPU acceleration more accessible to the developers that can benefit from it. AMD's new cross-platform development ecosystem ROCm provides promise for developing applications and solutions that work across systems running both AMD and non-AMD GPU computing hardware.

\quad This thesis presents Millipyde, a framework for GPU acceleration in Python using AMD's ROCm. Millipyde includes two new types, the \verb|gpuarray| and \verb|gpuimage|, as well as three new constructs for building GPU-accelerated applications -- the Operation, Pipeline, and Generator. Using these tools, Millipyde hopes to make it easier for engineers and researchers to write GPU-accelerated code in Python. Millipyde also has the potential to schedule work across many GPUs in complex multi-device environments. These capabilities will be demonstrated in a sample application of augmenting images on-device for machine learning applications. Our results showed that Millipyde is capable of making individual image-related transformations up to around 200 times faster than their CPU-only equivalents. Constructs such as the Millipyde's Pipeline was also able to additionally improve performance in certain situations, and it performed best when it was allowed to transparently schedule work across multiple devices. 
